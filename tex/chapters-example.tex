%=========================================================================
% (c) Michal Bidlo, Bohuslav Křena, 2008

\chapter{Introduction}
In~the~software industry, the~great and~accessible code is crucial for modern
business, and~the~best way for us to~connect and~share it is through APIs. But
not only there hasn't been an~industry standard for~designing APIs, there hasn't
been an~industry standard for documenting them. APIs are supposed to~connect
engineers and~allow for~the~sharing of~great developments. APIs let companies
add value to~the~products and~create an~ecosystem of~shared knowledge. But
an~API is only valuable if~it's accessible. And for that, it needs clear, human
and~machine readable documentation. So developers have worked hard to~find a~way
to~present their APIs.

This thesis aim to~solve the~problem of~presenting APIs. The~goal is to~design
an~application that provides innovative user interface with regard to~clarity
and~simple use, aimed to~developers, even in~case of~large remote interfaces.
The~application will be based on~existing JRAPIDoc \cite{JRAPIDoc} tool that
provides a~way to~automate API code generation, but currently lacks its good
represenetation. Ultimately, the~improvements should provide not~only
a~remote API reference, but also the functionality to~test and~call the~listed
web services.

The thesis is organized as~follows. Chapter \ref{preliminaries} gives
definitions needed to~follow the~thesis and~provides overview of~an~existing API
reference generators. Chapter \ref{technology} focuses on~technologies that are
going to~be used for~building the~improvements. Chapter
\ref{design} describes the~iterative application design that is based on~live
mockups and~later~on~PatternFly Framework. The last Chapter contains an~overall summary
of~the~designed solution and~final thoughts on~the~work done within this thesis.

\chapter{Preliminaries And Definitions}
\label{preliminaries}
This chapter will gradually introduce terms necessary to~follow the~thesis.
In the first sections, it will provide explanation why is API documentation
important and~why it is important to~automate API code generation. In~the~next
section, we will go over the~basic terminology and~establish notion of~remote
interfaces. The~last section covers the existing solutions for~API code
generation, their advantages and~disadvantages.

\section{What is API Documentation and Why It Matters?}
In~this section we explain what API is, we introduce the~basic terms that relate
to~APIs and~we focus on~why it is important to~document APIs.

\subsection{What is Application Programming Interface?}
In~computer programming an application programming interface (API) is
the~defined interface through which interactions happen between an~enterprise
and~users of~its assets. It can become the~primary entry point for~enterprise
services, for~its own website and~applications, as~well as~for~a~partner and~customer integrations. APIs are
defined through a~contract so~that any application can use it with relative
ease. It is an~architectural approach that resolves around
providing programmable interfaces to~a~set of~services to~different applications
serving different kinds of~customers. It assumes that these different user
groups might change or~evolve over time in~the way they utilize services.
The~point is to~create a~loosely coupled architecture that allows a~component
service to~have a~wide range of~future uses, and is technology agnostic.
This strategy leads to~the~following benefits:

\begin{itemize}
  \item \textbf{Automation}. With APIs, computers rather than people can manage
  the work. Through APIs, agencies can update work flows to~make them quicker
  and~more productive.
  \item \textbf{Integration}. APIs allow content to~be embedded from any site
  or~application more easily. This guarantees more fluid information delivery
  and~an~integrated user experience.
  \item \textbf{Personalization}. Through APIs any user or~company can customize
  the~content and~services that they use the~most.
  \item \textbf{Reduction of costs}. APIs are a~cheaper way of~building
  applications by~increasing the~reuse of~services. Providing a~usage or~\uv{analytics-based}
  evolutionary development platform decreases cost of~development and~change
  to~services.
  \item \textbf{Increasing customer loyalty}. The~company who releases
  the~API allows its customers to~access their conferencing services in~new,
  more efficient ways, increasing brand recognition and~customer loyalty.
\end{itemize}

Overall, APIs allow users to~enhance and~add services over existing products.
This introduces the~product to~a~new type of~user: \uv{the~third-party}
developer. However, catering to~the~developers is tricky. They are analytical,
precise, and are trying to~solve important problems with your API. They want to
understand how to~use the API effectively, which is where API documentation
comes into the~picture.

\subsection{What is API Documentation and Why We Need One?}
API documentation is a~technical content deliverable, containing instructions
about how to~effectively use and~integrate with~an~API. It is a~concise
reference manual that contains all the~information required to~work
with~the~API, with~details about the~functions, classes, return types and~more,
supported by~tutorials and~examples. 

The~\uv{thid-party} developers, who are API's main consumers, are busy solving
complex programming challenges. They want to~integrate as~quickly as~possible to~move forward in~their
software development, meaning they should immediately understand the~value
and~usage of~the~API. The~aggregate experience of~the~developer when
discovering, learning to~use, and~finally integrating with the API is termed
as~Developer Experience (DX) \cite{practical-api}.  Generally speaking, DX is
essentially the~sum of~the~experiences using the~API. If many of~those experiences were difficult
or~just annoying, users are going to~have a~bad experience and~they won't want
to~use the~product that the~API provides. Therefore good API documentation is
the~key to~a~great DX. You can have the~best, functional product, but no one
will use it if they don't know how~to. Documentation is the~foundation for~good
Developer Experience. Following lists the~exact reasons, why API documentation
matters.

// TODO pridat poznamku pod carou, co to je network effect
\begin{itemize}
  \item \textbf{Increased Awareness}. The~network effect is the~phenomenon when
  a~service or~product becomes more valuable when more people use~it. Satisfied
  consumers will be the~APIs biggest advocates. As~more users adopt the~API
  and~reach critical mass, there will be a~probable increase in~publicity,
  leading to~the~network effect. 
  \item \textbf{Easier maintenance}. Good documentation leads directly to~good
  product maintenance. It helps internal teams know the details of~resources,
  methods, and their associated requests and~responses, making maintenance
  and~updates quicker.
  \item \textbf{Saves time and costs}. Good documentation decreases
  the~amount of~time spent onboarding new users. In~addition, if API provides
  the~ability to~try out the API before implementing it, the~amount of~time
  spent onboarding will decrease even more.
\end{itemize}

The~point is that documentation is the~key to~a~great experience when consuming
the~API.

%// TODO link  "API-fication" (PDF download). www.hcltech.com. August 2014. ->
% file:///C:/Users/toucheqt/Downloads/apis_for_dsi.pdf

\subsection{Why automate API code generation?}
Among all the~phases in~the~API lifecycle, documentation is probably the~area
showing the~most growth. However the~developers pay little to~no attention it
when launching code. For a~lot of~developers it is because writing documentation
is still tedious and~time consuming task. This is especially true for teams that
rely on~static documentation that is manually updated with every release of~new
version of~the~API. Therefore the~more we can automate around the~documentation,
the~less work it is for the~developers to~iterate and~make improvements.

That is why there is great need of~a~framework that allows developers and~teams
to~design, build, document and~consume their APIs with primary focus
on~the~ability to~generate interactive documentation. This documentation allows
anyone to~visualize and~interact with the~API's resources without having any
implementation logic in~place.

The~\uv{auto-generated} documentation is a~central resource that the~developers
and~teams can customize, and build on~to~create a~more comprehensive user manual
for working with the~API.

\section{Understanding Web Services}
A~web service is a~software system designed to~support interoperable machine
to~machine interaction over a~network.
A~web service is a~collection of~open protocols and~standards used for
exchanging data between applications or~systems. Software
applications written in~various programming languages and~running on~various
platforms can use web services to~exchange data over computer networks like
the~Internet in~a~manner similar to~interprocess communication on~a~single
computer. This interoperability is due~to~the~use of~open standards. 

// TODO link na soap
In~the~past, web services used mostly SOAP over HTTP protocol, allowing less
costly interactions over the~internet than via proprietary solutions like
EDI/B2B. In~a~2004, the W3C extended the~definition of~web services about
REST-complient web services, in~which the~primary purpose of~the web service is
to~manipulate XML representations of~web resources using a~uniform set
of~stateless operations.

\subsection{\uv{SOAP-based} Web Services}
SOAP is an~acronym for Simple Object Access Protocol. It's a~\uv{XML-based}
messaging protocol for~exchanging information among computers that uses WSDL
TODO poznamka pod carou wsdl is an XML-based protocol  that describes
how to access a web service and what operation it will perform\ref{wsdl}
for~communication between consumer and~provider of~the~web service.
Although SOAP can be used in a~variety of~messaging systems and~can be delivered
via~a~variety of~transport protocols, the~initial focus of~SOAP is remote procedure calls
transported via~HTTP.

A~SOAP message is an~ordinary XML document containing the~following elements:

\begin{itemize}
  \item \textbf{Envelope}. Defines the~start and~the~end of~the~message.
  Envelope is a~mandatory element.
  \item \textbf{Header}. Contains any optional attributes of~the~message used
  in~processing the~message, either at~an~intermediary point or~at~the~ultimate
  \uv{end-point}. Header is an~optional element.
  \item \textbf{Body}. Contains the~XML data comprising the~message being sent.
  Body is a~mandatory element.
  \item \textbf{Fault}. An~optional Fault element that provides information
  about errors that occur while processing the~message.
\end{itemize}

As~shown in~listing \ref{soapExample}, a~SOAP message consists of~an~Envelope
element, inside which a~Header and a~Body element can be nested. Inside the~Body
element a~Fault element can be nested, if an~error occurs in~the~web service.
This SOAP message format can be used both to~send requests from the~client
to~the~web service, and~to~send responses back to~the~client from~the~web
service. Thus the~SOAP request and response message format is the same unlike
the~HTTP protocol where the~request and response formats are different.

\begin{lstlisting}[caption=Example of~SOAP Message Structure., language=XML,
label=soapExample]
	<?xml version="1.0"?>
	<soap:Envelope xmlns:soap="http://www.w3.org/2001/12/soap-envelope">
		<soap:Header>
			...
		</soap:Header>
		<soap:Body>
			<!-- Fault element is optional,
				 used only if a fault occurs in a web service -->
			<soap:Fault>
				...
			</soap:Fault>
		</soap:Body>
	</soap:Envelope>		
\end{lstlisting}

When a~client and~a~web service communicate via SOAP, they exchange messages.
In~the~SOAP specification \ref{soap} are described two basic message exchange
patterns.

\begin{itemize}
  \item Request - Response
  \item Response
\end{itemize}

\subsubsection{Request - Response pattern}
In~a~request - response message exchange the SOAP client sends a~SOAP request
message to~the~web service. The~service responds with a~SOAP response message.

A~request - response message exchange pattern is necessary when the~SOAP client
needs to~send data to~the~SOAP service. For~instance, if the~client request that
data be stored by~the~service, the client needs to~send~the~data to~be stored
to~the~service.

// TODO obrazek example

\subsubsection{Response pattern}
In~a~response message exchange the~SOAP client connects to~the~service, but does
not send a~SOAP message through. It just sends a~simple HTTP request
for~instance. The~service responds with a~SOAP message. This pattern is similar
to~how browser communicates with a~web server. The browser sends an~HTTP request
telling what page it wants to~access, but the HTTP request does not carry any
additional data. The~web server responds with an~HTTP response carrying
the~requested page.

// TODO obrazek example



\subsection{RESTful Web Services}

 https://en.wikipedia.org/wiki/Web_service
https://www.service-architecture.com/articles/web-services/web_services_explained.html
https://www.tutorialspoint.com/webservices/what_are_web_services.htm
https://www.javatpoint.com/what-is-web-service

https://tools.ietf.org/html/draft-box-http-soap-00 SOAP RFC

\section{Existing solutions for API code generation}
\subsection{JRAPIDoc}
\subsection{Swagger}
\subsection{Apiary}
\subsection{Spring REST Docs}
\subsection{Postman}
https://nordicapis.com/ultimate-guide-to-30-api-documentation-solutions/

\chapter{Technologic overview}
\label{technology}
\section{Angular}
\section{TypeScript}
\section{PatternFly}

\chapter{Application Design}
\label{design}
\section{The initial state}
\section{Building live mockups}
\section{Improving the mockups with PatternFly}

\chapter{Conclusion}
https://blog.readme.io/what-is-swagger-and-why-it-matters/

\begin{itemize}
  \item Technologies for Remote Interfaces
  \begin{itemize}
    \item SOAP-based Web Services
    \item RESTful Web Services
  \end{itemize}
  \item What are the existing tools for Remote API Web Reference?
  \begin{itemize}
    \item Swagger
    \item SoapUI
  \end{itemize}
  \item 
\end{itemize}

\begin{itemize}
\item Musíme mít co říci,
\item musíme vědět, komu to chceme říci,
\item musíme si dokonale promyslet obsah,
\item musíme psát strukturovaně. 
\end{itemize}

Tyto a další pokyny jsou dostupné též na školních internetových stránkách \cite{fitWeb}.

Přehled základů typografie a tvorby dokumentů s využitím systému \LaTeX je 
uveden v~\cite{Rybicka}.

\section{Musíme mít co říci}
Dalším důležitým předpokladem dobrého psaní je {\it psát pro někoho}. Píšeme-li si poznámky sami pro sebe, píšeme je jinak než výzkumnou zprávu, článek, diplomovou práci, knihu nebo dopis. Podle předpokládaného čtenáře se rozhodneme pro způsob psaní, rozsah informace a~míru detailů.

\section{Musíme vědět, komu to chceme říci}
Dalším důležitým předpokladem dobrého psaní je psát pro někoho. Píšeme-li si poznámky sami pro sebe, píšeme je jinak než výzkumnou zprávu, článek, diplomovou práci, knihu nebo dopis. Podle předpokládaného čtenáře se rozhodneme pro způsob psaní, rozsah informace a~míru detailů.

\section{Musíme si dokonale promyslet obsah}
Musíme si dokonale promyslet a~sestavit obsah sdělení a~vytvořit pořadí, v~jakém chceme čtenáři své myšlenky prezentovat. 
Jakmile víme, co chceme říci a~komu, musíme si rozvrhnout látku. Ideální je takové rozvržení, které tvoří logicky přesný a~psychologicky stravitelný celek, ve kterém je pro všechno místo a~jehož jednotlivé části do sebe přesně zapadají. Jsou jasné všechny souvislosti a~je zřejmé, co kam patří.

Abychom tohoto cíle dosáhli, musíme pečlivě organizovat látku. Rozhodneme, co budou hlavní kapitoly, co podkapitoly a~jaké jsou mezi nimi vztahy. Diagramem takové organizace je graf, který je velmi podobný stromu, ale ne řetězci. Při organizaci látky je stejně důležitá otázka, co do osnovy zahrnout, jako otázka, co z~ní vypustit. Příliš mnoho podrobností může čtenáře právě tak odradit jako žádné detaily.

Výsledkem této etapy je osnova textu, kterou tvoří sled hlavních myšlenek a~mezi ně zařazené detaily.

\section{Musíme psát strukturovaně} 
Musíme začít psát strukturovaně a~současně pracujeme na co nejsrozumitelnější formě, včetně dobrého slohu a~dokonalého značení. 
Máme-li tedy myšlenku, představu o~budoucím čtenáři, cíl a~osnovu textu, můžeme začít psát. Při psaní prvního konceptu se snažíme zaznamenat všechny své myšlenky a~názory vztahující se k~jednotlivým kapitolám a~podkapitolám. Každou myšlenku musíme vysvětlit, popsat a~prokázat. Hlavní myšlenku má vždy vyjadřovat hlavní věta a~nikoliv věta vedlejší.

I k~procesu psaní textu přistupujeme strukturovaně. Současně s~tím, jak si ujasňujeme strukturu písemné práce, vytváříme kostru textu, kterou postupně doplňujeme. Využíváme ty prostředky DTP programu, které podporují strukturovanou stavbu textu (předdefinované typy pro nadpisy a~bloky textu). 


\chapter{Několik formálních pravidel}
Naším cílem je vytvořit jasný a~srozumitelný text. Vyjadřujeme se proto přesně, píšeme dobrou češtinou (nebo zpravidla angličtinou) a~dobrým slohem podle obecně přijatých zvyklostí. Text má upravit čtenáři cestu k~rychlému pochopení problému, předvídat jeho obtíže a~předcházet jim. Dobrý sloh předpokládá bezvadnou gramatiku, správnou interpunkci a~vhodnou volbu slov. Snažíme se, aby náš text nepůsobil příliš jednotvárně používáním malého výběru slov a~tím, že některá zvlášť oblíbená slova používáme příliš často. Pokud používáme cizích slov, je samozřejmým předpokladem, že známe jejich přesný význam. Ale i~českých slov musíme používat ve správném smyslu. Např. platí jistá pravidla při používání slova {\it zřejmě}. Je {\it zřejmé} opravdu zřejmé? A~přesvědčili jsme se, zda to, co je {\it zřejmé} opravdu platí? Pozor bychom si měli dát i~na příliš časté používání zvratného se. Například obratu {\it dokázalo se}, že\ldots{} zásadně nepoužíváme. Není špatné používat autorského {\it my}, tím předpokládáme, že něco řešíme, nebo například zobecňujeme spolu se čtenářem. V~kvalifikačních pracích použijeme autorského {\it já} (například když vymezujeme podíl vlastní práce vůči převzatému textu), ale v~běžném textu se nadměrné používání první osoby jednotného čísla nedoporučuje.

Za pečlivý výběr stojí i~symbolika, kterou používáme ke {\it značení}. Máme tím na mysli volbu zkratek a~symbolů používaných například pro vyjádření typů součástek, pro označení hlavních činností programu, pro pojmenování ovládacích kláves na klávesnici, pro pojmenování proměnných v~matematických formulích a~podobně. Výstižné a~důsledné značení může čtenáři při četbě textu velmi pomoci. Je vhodné uvést seznam značení na začátku textu. Nejen ve značení, ale i~v~odkazech a~v~celkové tiskové úpravě je důležitá důslednost.

S tím souvisí i~pojem z~typografie nazývaný {\it vyznačování}. Zde máme na mysli způsob sazby textu pro jeho zvýraznění. Pro zvolené značení by měl být zvolen i~způsob vyznačování v~textu. Tak například klávesy mohou být umístěny do obdélníčku, identifikátory ze~zdrojového textu mohou být vypisovány {\tt písmem typu psací stroj} a~podobně.

Uvádíme-li některá fakta, neskrýváme jejich původ a~náš vztah k~nim. Když něco tvrdíme, vždycky výslovně uvedeme, co z~toho bylo dokázáno, co teprve bude dokázáno v~našem textu a~co přebíráme z~literatury s~uvedením odkazu na příslušný zdroj. V~tomto směru nenecháváme čtenáře nikdy na pochybách, zda jde o~myšlenku naši nebo převzatou z~literatury.

Nikdy neplýtváme čtenářovým časem výkladem triviálních a~nepodstatných informací. Neuvádíme rovněž několikrát totéž jen jinými slovy. Při pozdějších úpravách textu se nám může některá dříve napsaná pasáž jevit jako zbytečně podrobná nebo dokonce zcela zbytečná. Vypuštění takové pasáže nebo alespoň její zestručnění přispěje k~lepší čitelnosti práce! Tento krok ale vyžaduje odvahu zahodit čas, který jsme jejímu vytvoření věnovali. 


\chapter{Nikdy to nebude naprosto dokonalé}
Když jsme už napsali vše, o~čem jsme přemýšleli, uděláme si den nebo dva dny volna a~pak si přečteme sami rukopis znovu. Uděláme ještě poslední úpravy a~skončíme. Jsme si vědomi toho, že vždy zůstane něco nedokončeno, vždy existuje lepší způsob, jak něco vysvětlit, ale každá etapa úprav musí být konečná.


\chapter{Typografické a~jazykové zásady}
Při tisku odborného textu typu {\it technická zpráva} (anglicky {\it technical report}), ke kterému patří například i~text kvalifikačních prací, se často volí formát A4 a~často se tiskne pouze po~jedné straně papíru. V~takovém případě volte levý okraj všech stránek o~něco větší než pravý -- v~tomto místě budou papíry svázány a~technologie vazby si tento požadavek vynucuje. Při vazbě s~pevným hřbetem by se levý okraj měl dělat o~něco širší pro tlusté svazky, protože se stránky budou hůře rozevírat a~levý okraj se tak bude oku méně odhalovat.

Horní a~spodní okraj volte stejně veliký, případně potištěnou část posuňte mírně nahoru (horní okraj menší než dolní). Počítejte s~tím, že při vazbě budou okraje mírně oříznuty.

Pro sazbu na stránku formátu A4 je vhodné používat pro základní text písmo stupně (velikosti) 11 bodů. Volte šířku sazby 15 až 16 centimetrů a~výšku 22 až 23 centimetrů (včetně případných hlaviček a~patiček). Proklad mezi řádky se volí 120 procent stupně použitého základního písma, což je optimální hodnota pro rychlost čtení souvislého textu. V~případě použití systému LaTeX ponecháme implicitní nastavení. Při psaní kvalifikační práce se řiďte příslušnými závaznými požadavky.

Stupeň písma u~nadpisů různé úrovně volíme podle standardních typografických pravidel. 
Pro všechny uvedené druhy nadpisů se obvykle používá polotučné nebo tučné písmo (jednotně buď všude polotučné nebo všude tučné). Proklad se volí tak, aby se následující text běžných odstavců sázel pokud možno na {\it pevný rejstřík}, to znamená jakoby na linky s~předem definovanou a~pevnou roztečí.

Uspořádání jednotlivých částí textu musí být přehledné a~logické. Je třeba odlišit názvy kapitol a~podkapitol -- píšeme je malými písmeny kromě velkých začátečních písmen. U~jednotlivých odstavců textu odsazujeme první řádek odstavce asi o~jeden až dva čtverčíky (vždy o~stejnou, předem zvolenou hodnotu), tedy přibližně o~dvě šířky velkého písmene M základního textu. Poslední řádek předchozího odstavce a~první řádek následujícího odstavce se v~takovém případě neoddělují svislou mezerou. Proklad mezi těmito řádky je stejný jako proklad mezi řádky uvnitř odstavce.

Při vkládání obrázků volte jejich rozměry tak, aby nepřesáhly oblast, do které se tiskne text (tj. okraje textu ze všech stran). Pro velké obrázky vyčleňte samostatnou stránku. Obrázky nebo tabulky o~rozměrech větších než A4 umístěte do písemné zprávy formou skládanky všité do přílohy nebo vložené do záložek na zadní desce.

Obrázky i~tabulky musí být pořadově očíslovány. Číslování se volí buď průběžné v~rámci celého textu, nebo -- což bývá praktičtější -- průběžné v~rámci kapitoly. V~druhém případě se číslo tabulky nebo obrázku skládá z~čísla kapitoly a~čísla obrázku/tabulky v~rámci kapitoly -- čísla jsou oddělena tečkou. Čísla podkapitol nemají na číslování obrázků a~tabulek žádný vliv.

Tabulky a~obrázky používají své vlastní, nezávislé číselné řady. Z toho vyplývá, že v~odkazech uvnitř textu musíme kromě čísla udat i~informaci o~tom, zda se jedná o~obrázek či tabulku (například \uv{\ldots {\it viz tabulka 2.7} \ldots}). Dodržování této zásady je ostatně velmi přirozené.

Pro odkazy na stránky, na čísla kapitol a~podkapitol, na čísla obrázků a~tabulek a~v~dalších podobných příkladech využíváme speciálních prostředků DTP programu, které zajistí vygenerování správného čísla i~v~případě, že se text posune díky změnám samotného textu nebo díky úpravě parametrů sazby. Příkladem takového prostředku v~systému LaTeX je odkaz na číslo odpovídající umístění značky v~textu, například návěští ($\backslash${\tt ref\{navesti\}} -- podle umístění návěští se bude jednat o~číslo kapitoly, podkapitoly, obrázku, tabulky nebo podobného číslovaného prvku), na stránku, která obsahuje danou značku ($\backslash${\tt pageref\{navesti\}}), nebo na literární odkaz ($\backslash${\tt cite\{identifikator\}}).

Rovnice, na které se budeme v~textu odvolávat, opatříme pořadovými čísly při pravém okraji příslušného řádku. Tato pořadová čísla se píší v~kulatých závorkách. Číslování rovnic může být průběžné v~textu nebo v~jednotlivých kapitolách.

Jste-li na pochybách při sazbě matematického textu, snažte se dodržet způsob sazby definovaný systémem LaTeX. Obsahuje-li vaše práce velké množství matematických formulí, doporučujeme dát přednost použití systému LaTeX.

Mezeru neděláme tam, kde se spojují číslice s~písmeny v~jedno slovo nebo v~jeden znak -- například {\it 25krát}.

Členicí (interpunkční) znaménka tečka, čárka, středník, dvojtečka, otazník a~vykřičník, jakož i~uzavírací závorky a~uvozovky se přimykají k~předcházejícímu slovu bez mezery. Mezera se dělá až za nimi. To se ovšem netýká desetinné čárky (nebo desetinné tečky). Otevírací závorka a~přední uvozovky se přimykají k~následujícímu slovu a~mezera se vynechává před nimi -- (takto) a~\uv{takto}.

Pro spojovací a~rozdělovací čárku a~pomlčku nepoužíváme stejný znak. Pro pomlčku je vyhrazen jiný znak (delší). V~systému TeX (LaTeX) se spojovací čárka zapisuje jako jeden znak \uv{pomlčka} (například \uv{Brno-město}), pro sázení textu ve smyslu intervalu nebo dvojic, soupeřů a~podobně se ve zdrojovém textu používá dvojice znaků \uv{pomlčka} (například \uv{zápas Sparta -- Slavie}; \uv{cena 23--25 korun}), pro výrazné oddělení části věty, pro výrazné oddělení vložené věty, pro vyjádření nevyslovené myšlenky a~v~dalších situacích (viz Pravidla českého pravopisu) se používá nejdelší typ pomlčky, která se ve zdrojovém textu zapisuje jako trojice znaků \uv{pomlčka} (například \uv{Další pojem --- jakkoliv se může zdát nevýznamný --- bude neformálně definován v~následujícím odstavci.}). Při sazbě matematického mínus se při sazbě používá rovněž odlišný znak. V~systému TeX je ve zdrojovém textu zapsán jako normální mínus (tj. znak \uv{pomlčka}). Sazba v~matematickém prostředí, kdy se vzoreček uzavírá mezi dolary, zajistí vygenerování správného výstupu.

Lomítko se píše bez mezer. Například školní rok 2008/2009.

Pravidla pro psaní zkratek jsou uvedena v~Pravidlech českého pravopisu \cite{Pravidla}. I~z~jiných důvodů je vhodné, abyste tuto knihu měli po ruce. 


\section{Co to je normovaná stránka?}
Pojem {\it normovaná stránka} se vztahuje k~posuzování objemu práce, nikoliv k~počtu vytištěných listů. Z historického hlediska jde o~počet stránek rukopisu, který se psal psacím strojem na speciální předtištěné formuláře při dodržení průměrné délky řádku 60 znaků a~při 30 řádcích na stránku rukopisu. Vzhledem k~zápisu korekturních značek se používalo řádkování 2 (ob jeden řádek). Tyto údaje (počet znaků na řádek, počet řádků a~proklad mezi nimi) se nijak nevztahují ke konečnému vytištěnému výsledku. Používají se pouze pro posouzení rozsahu. Jednou normovanou stránkou se tedy rozumí 60*30 = 1800 znaků. Obrázky zařazené do textu se započítávají do rozsahu písemné práce odhadem jako množství textu, které by ve výsledném dokumentu potisklo stejně velkou plochu.

Orientační rozsah práce v~normostranách lze v~programu Microsoft Word zjistit pomocí funkce {\it Počet slov} v~menu {\it Nástroje}, když hodnotu {\it Znaky (včetně mezer)} vydělíte konstantou 1800. Do rozsahu práce se započítává pouze text uvedený v~jádru práce. Části jako abstrakt, klíčová slova, prohlášení, obsah, literatura nebo přílohy se do rozsahu práce nepočítají. Je proto nutné nejdříve označit jádro práce a~teprve pak si nechat spočítat počet znaků. Přibližný rozsah obrázků odhadnete ručně. Podobně lze postupovat i~při použití OpenOffice. Při použití systému LaTeX pro sazbu je situace trochu složitější. Pro hrubý odhad počtu normostran lze využít součet velikostí zdrojových souborů práce podělený konstantou cca 2000 (normálně bychom dělili konstantou 1800, jenže ve zdrojových souborech jsou i~vyznačovací příkazy, které se do rozsahu nepočítají). Pro přesnější odhad lze pak vyextrahovat holý text z~PDF (např. metodou cut-and-paste nebo {\it Save as Text\ldots}) a~jeho velikost podělit konstantou 1800. 


\chapter{Závěr}
Závěrečná kapitola obsahuje zhodnocení dosažených výsledků se zvlášť vyznačeným vlastním přínosem studenta. Povinně se zde objeví i zhodnocení z pohledu dalšího vývoje projektu, student uvede náměty vycházející ze zkušeností s řešeným projektem a uvede rovněž návaznosti na právě dokončené projekty.

%=========================================================================
